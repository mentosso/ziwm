\documentclass[11pt, a4paper, titlepage]{report}
\linespread{1.2}
\usepackage[polish]{babel}
\usepackage[utf8]{inputenc}
\usepackage{polski}
\usepackage[T1]{fontenc}
\usepackage{graphicx}
\usepackage{enumitem}
\usepackage[hidelinks]{hyperref}
\usepackage{afterpage}
\usepackage{listings}
\usepackage{color}
\frenchspacing
\addtolength{\textwidth}{3.5cm}
\addtolength{\hoffset}{-2cm}
\addtolength{\textheight}{3.5cm}
\addtolength{\voffset}{-2cm}
\usepackage{indentfirst}
\usepackage{caption}
\definecolor{mygreen}{rgb}{0,0.6,0}
\definecolor{mygray}{rgb}{0.5,0.5,0.5}
\definecolor{mymauve}{rgb}{0.58,0,0.82}

\renewcommand\lstlistlistingname{Spis listingów}
\lstset{ %
  language=Matlab,
	backgroundcolor=\color{white},   % choose the background color
	basicstyle=\footnotesize,        % size of fonts used for the code
	breaklines=true,                 % automatic line breaking only at whitespace
	frame=single,
	commentstyle=\color{mygreen},    % comment style
	escapeinside={\%*}{*)},          % if you want to add LaTeX within your code
	keywordstyle=\color{blue},       % keyword style
	stringstyle=\color{mymauve},     % string literal style
}
\date{Wrocław, 06.06.2016}
\makeatletter
\renewcommand{\maketitle}{\begin{titlepage}
		\begin{center}\small
			Politechnika Wrocławska\\
			Wydział Elektroniki\\
			Rok akad. 2015/2016\\
			Kierunek Informatyka\\
			\vspace{3cm}
			\rule{\linewidth}{0.4pt}
				\huge \textsc{\textbf{Komputerowe wspomaganie diagnozowania choroby niedokrwiennej u dzieci z wykorzystaniem algorytmów minimalno-odległościowych}}
				\vspace{0.5cm} \\
				\normalsize \textit{Zastosowanie informatyki w medycynie: Projekt}
			\rule{\linewidth}{0.4pt}
		\end{center}

		\vspace{3cm}
		\begin{flushleft}
			\textbf{\textit{Grupa projektowa:}} \hspace{6.5cm} \textbf{\textit{Prowadzący:}} \\
			Barłomiej Grzegorek, XXXXXX \hfill{prof. dr hab. inż. Marek Kurzyński} \\
			Marcin Mantke, 200633\\
			\vspace{2cm}
		\end{flushleft}
		\vspace*{\stretch{6}}
		\begin{center}
			\@date
		\end{center}
	\end{titlepage}%
}
\makeatother
\begin{document}
\maketitle
\tableofcontents
\cleardoublepage
\phantomsection
\addcontentsline{toc}{chapter}{\listfigurename}
\listoffigures

\cleardoublepage
\phantomsection
\addcontentsline{toc}{chapter}{Spis listingów}
\lstlistoflistings
\chapter{Charakterystyka analizowanego problemu}
\label{chap:Charakterystyka analizowanego problemu}

\chapter{Opis stosowanych algorytmów}
\label{chap:Opis stosowanych algorytmów}

\chapter{Informacja o środowisku implementacyjnym}
\label{chap:Informacja o środowisku implementacyjnym}

\chapter{Opis badań eksperymentalnych}
\label{chap:Opis badań eksperymentalnych}

\chapter{Wyniki badań}
\label{chap:Wyniki badań}

\chapter{Podsumowanie i wnioski}
\label{chap:Podsumowanie i wnioski}
\section{Wnioski płynące z analizy wyników}
\section{Ocena krytyczna i podsumowanie projektu}

\begin{thebibliography}{}
\addcontentsline{toc}{chapter}{Bibliografia}
\bibitem{1}
M.M. Sysło, N.Deo, J.S. Kowalik, \textit{Algorytmy optymalizacji dyskretnej z programami w języku Pascal}, Wydawnictwo Naukowe PWN , 1999

\bibitem{2}
R. Neapolitan, K. Naimipour, \textit{Podstawy Algorytmów z przykładami w C++}, Helion, 2004

\bibitem{3}
T. H. Cormen, C. E. Leiserson, R. L. Rivest, C. Stein, \textit{Wprowadzenie do algorytmów}, Wydawnictwo Naukowe PWN, 2013

\end{thebibliography}
\end{document}
